%% ---------------------------------------------------------------------------
%% Für das Zeichnen der unterschiedlichen Sektionen, Kapitel ...
%%
%% Datei:   bookstyle.tex
%%
%% Datum:   18.03.2018
%%
%% Author:  paule32 - J. Kallup
%%
%% Lizenz;  MIT
%%----------------------------------------------------------------------------

%----------------------------------------------------------------
% Eine Sektion kann ein Kapitel oder auch Untergrupp(en) sein,
% es wird eine Neue Seite erstellt und sämtliche Informationen
% werden gelöscht. Informationen über die aktuelle Seite
% werden mit \WennSektion automatisch gesetzt ...
%----------------------------------------------------------------
\newcommand{\sectiontitle}{}
\newcommand{\newsection}[1]{%
\section*{}\renewcommand{\sectiontitle}{#1}}

\setlength{\parindent}{0em}   % kein Absatz-Einzug

% nummerierte Kapitel:
\renewcommand{\chaptermark}[1]{%
\markboth{\chapapp\ \thechapter\autodot}{#1}}

% Kopfzeile
\newcommand{\MyStyleBook}[2]{%
  \thispagestyle{empty}
  \begin{tikzpicture}[overlay, remember picture]
    % Header
    \fill[cyan]   (-15.2, 3.0)
    rectangle     ( 5.99, 0.2); % Hintergrund
    \fill[orange] (-15.2, 3.0)
    rectangle     ( 5.99, 2.6); % Stripe
    %
    \draw [black] (#2, 1.0) node {\Huge{\textbf{#1}}};
    %
    \draw [yellow] (4, 1.2) node { % römische Ziffern für
        \pagenumbering{Roman}      % die Beschriftung
        \Huge{
        \textbf{
        \thepage}}
        \pagenumbering{arabic}     % arabische Ziffern 0-9
    };
    % Footer
    \fill[cyan] (-15.2, -25.8 ) rectangle   (  5.99, -27.2 );
    %
    \draw [black] (-12.2, -26.25) node {\Large{\textbf{de.sci.mathematik}}};
    %
    \fill[orange] (-15.2, -26.99 ) rectangle   (  5.99, -26.71 );
        \draw [yellow] (4, -26.25) node { % römische Ziffern für
        \pagenumbering{Roman}      % die Beschriftung
        \Large{\textbf{
        \thepage}}
        \pagenumbering{arabic}};   % arabische Ziffern 0-9
  \end{tikzpicture}
}%

%-----------------------------------------------
% if Abfragen für Header/Footer Layout ...
%-----------------------------------------------
\newcommand{\WennSektion}[2]{
  \IfStrEq{\sectiontitle}{#1}{        % wenn true, dann ...
    \ifthispageodd{                    % wenn Seite nicht gerade, dann ...
      \MyStyleBook{#1}{#2}            % Arg1: Name, Arg2: PositionTuning
    }{                                % ansonsten gerade, dann ...
    i am gerade
    }
  }{}                          % wenn false, dann nix
}
%---------------------
\def\chopline#1;#2 \\{
  \def\name{#1}
  \def\position{#2}
}
\newif\ifmore \moretrue
%--------------------------
\newcommand{\MyReadTopics}{%
\newread\quelle
\openin\quelle=helpfile.tmp
\loop
  \read\quelle to \zeile
  \ifeof\quelle
    \global\morefalse
  \else
    \expandafter\chopline\zeile\\
    {\name}{\position}              % <-- hier header/footer ??
  \fi
\ifmore\repeat
\closein\quelle
}
%------------------------------
\pagestyle{scrheadings}
\clearscrheadfoot
\setkomafont{pagehead}{\small}
\ohead[\MyReadTopics]{}

% Fußzeile
\ofoot[\pagemark]{\pagemark}

