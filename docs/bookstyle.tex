%% ---------------------------------------------------------------------------
%% Für das Zeichnen der unterschiedlichen Sektionen, Kapitel ...
%%
%% Datei:   bookstyle.tex
%% Datum:   18.03.2018
%% Author:  paule32 - J. Kallup
%%
%% Lizenz;  MIT * non-profit use only!
%%----------------------------------------------------------------------------

%----------------------------------------------------------------
% Eine Sektion kann ein Kapitel oder auch Untergrupp(en) sein,
% es wird eine Neue Seite erstellt und sämtliche Informationen
% werden gelöscht. Informationen über die aktuelle Seite
% werden automatisch gesetzt ...
%----------------------------------------------------------------
\newcommand{\sectiontitle}{}
\newcommand{\newsection}[1]{%
\section*{}\renewcommand{\sectiontitle}{#1}}

%------------------------------------------
% die Maße des Buch-Layout's ...
%------------------------------------------
\setlength{\parindent}{0em} % kein Absatz-Einzug
\usepackage{geometry}
\geometry{
  a4paper,    % Standard Format - Deutschland
  right=0mm,  % rechter Rand bei 0mm
  left=0mm,   % links anfangen bei 0mm
  top=0mm,    % wir fangen bei 0mm oben an
  bottom=0mm, % und das Ende ist bei 0mm
}

%--------------------------------------------
% Alias-Namen ...
%--------------------------------------------
\def\MyChapterContents{Inhaltsverzeichnis}
\def\MyChapter{Kapitel}

\newcounter{MyPageCounter}
\setcounter{MyPageCounter}{1}

%--------------------------------------------
% römische Ziffern für Inhaltsverzeichnis ...
%--------------------------------------------
\newcommand{\MyPageNumbering}{%
\IfStrEq{\sectiontitle}{\MyChapterContents}{%
\Roman{MyPageCounter}}{% römische  Ziffern
\arabic{MyPageCounter}}% arabische Ziffern 0-9
}

%-----------------------------------------------------------------
% Kopfzeile + Fußzeile
%-----------------------------------------------------------------
\newcommand{\MyBookSide}[1]{%
  \newsection{#1}
  \newpage\thispagestyle{empty}
  %----------------------------------------------
  % den Header + Footer in eine Tabelle packen,
  % wegen einheitlichen maßen ...
  %----------------------------------------------
  \begin{figure}[t]
      \fboxsep0pt
      \colorbox{orange}{\begin{minipage}{20.94cm}
          \begin{tabular}{p{0.5cm}p{16.42cm}}&
          \small{(c) Jens Kallup - non-profit use only !!!} \\
          \end{tabular}
      \end{minipage}}
      \colorbox{cyan}{\begin{minipage}{20.94cm}
      \vspace{2mm}
      \begin{tabular}{p{0.5cm}p{16.42cm}c}
          &                  & {\Large\textcolor{yellow}{\textbf{\MyChapter}}} \\
          &{\Huge\textbf{#1}}& {\huge\textcolor{yellow}{\textbf{\MyPageNumbering}}}
      \end{tabular}
      \vspace{2mm}
      \end{minipage}}
      \vspace{-5mm}    % Absatz-Begin vom Text
  \end{figure}
  \begin{figure}[b]
      \fboxsep0pt
      \colorbox{cyan}{\begin{minipage}{20.94cm}
      \vspace{1mm}
      \begin{tabular}{p{0.5cm}p{18.2cm}c}
          &{\Large\textbf{de.sci.mathematik}}
          &{\Large\textbf{\textcolor{yellow}\MyPageNumbering}}\\
      \end{tabular}
      \vspace{1mm}
      \end{minipage}}
  \end{figure}
  %------------------------------------------------
  % anschließend wird schon der Seiten-Counter
  % hochgedreht ...
  %------------------------------------------------
  \stepcounter{MyPageCounter}
  \normalsize
}%


