\chapter{Grundlagen - Informatik}
\section{Grundlagen}
Die zum Erledigen einer Aufgabe erforderlichen Arbeitsschritte
lassen einen immer wiederkehrenden Rhytmus erkennen: \\
\textbf{E} - ingabe \\
\textbf{V} - erarbeitung \\
\textbf{A} - usgabe . \\
\\
Der Aufbau der Zentraleinheit aus elektronischen Schaltungen bringt
es mit sich, dass zur Datendarstellung nur zwei Zust"ande gegeben sind. \\
''Ja'' - Strom flie"sst und \\
''Nein'' - kein Strom. \\
Diese beiden Zust"ande werden mit \textbf{1} oder \textbf{0} bezeichnet.\\
Das Bit ist ein Bin"arzeichen, das die Zust"ande 1 und 0 annehmen kann.
Es ist zugleich die kleinste Informationseinheit, die Computer verstehen.
Alle Informationen m"ussen auf dem Bit aufgebaut werden.
\\
\\
Hinweis: F"ur Grundlagen der in der Inforamtik verwendeten Zahlensysteme,
finden Sie auch im Kapitel \ref{Mathegrundlagen} - \ref{Zahlensysteme}
auf Seite \pageref{Zahlensysteme} .

