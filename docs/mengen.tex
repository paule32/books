\chapter{Mengen}
Mengen k"onnen auf zwei verschiedene weisen dargestellt werden:
\begin{itemize}
  \item[1.]   \textbf{aufz"ahlende Schreibweise:}\\
  Alle Elemente einer Menge in geschweifter Klammer:
  $M = \{ 1;2;3 \}$ \\ \\
  Wenn in einer Menge ein l"angeres Intervall existiert,
  kann man sich durch Schreibweise \ldots bedienen.
  Diese Schreibweise kennzeichnet Elemente, die in der
  Menge *M* vorkommen k"onnen, jedoch aus Platzgründen
  nicht mit aufgeschrieben werden - man k"onnte es auch
  als Platzhalter verstehen. \\
  Und hier noch die Schreibweise: 
  $M = \{ 1;2;3; \ldots ;10;11 \}$
  \item[2.] \textbf{die beschreibende Schreibweise:} \\
  Mit dieser Schreibweise wird versucht, Elemente einer
  Menge mit mathematischen Aussagen zu beschreiben. Erf"ullt
  ein Element eine Aussage, so ist dieses Element der Menge: \\
  $M = \{ \: p \: | \: '' p \: ist \: eine \: Primzahl '' \: \}$ \\
  \\
  Sei Menge M eine mathematisch beschreibende Aussage:\\
  $M = \{z \in N \}$ \\
  \\
  dann spricht man von einer Menge M, in der ''z Element von N ist''. \\
  Wenn gilt: $M = \{z \le 17\}$\\
  \\
  dann spricht man von einer Menge, in der nur das Element z
  kleiner gleich 17 enthalten sein darf/ (oder alle Elemente
  von z kleiner gleich 17 sind). \\
\end{itemize}
Mengendiagramme sind Diagramme, die Elemente in einer
geschlossener Umgebung enthalten.

