\chapter{Zahlen und Zahlenbereiche}
\section{Nat"urliche Zahlen}
\subsection{Einf"uhrung}
Der Begriff Zahl ist vom althochdeutschen Wort ''zala'' abgeleitet.
Dieser Begriff wurde mit ''Einschnitt ins Kerbholz'' "ubersetzt.
Diese "Ubersetzung zeigt, welche Bedeutung der Zahlbegriff historisch hatte.
Er sollte helfen, die Welt messbar und z"ahlbar zu machen. Die V"olker mit einer
schriftlichen Kultur haben im Laufe der Geschichte verschiedene Zahlen- und Notationsysteme
entwickelt. \\
Unser heutiges Zahlensystem wird arabisch-indisches System genannt. Es basiert auf
dem Dezimalsystem und enth"alt die Ziffern 0 bis 9. Aus diesen Symbolen oder Zeichen lassen
sich nach einfachen Gesetzm"assigkeiten beliebige Zahlen bilden.
Die beiden Begriffe Kardinalzahlen und Ordinalzahlen beschreiben jeweils die T"atigket des
Ordnens und des Z"ahlens.
Unter Kardinalzahlen versteht man die ganzen Zahlen, mit denen gez"ahlt wird und Mengen
beschrieben werden. Beispielsweise spricht man von 2 Hunden, 4 Katzen, 32 Kilometer usw.
Die Ordinalzahlen hingegen beschreiben Ordnungen, Rang- und Reihenfolgen.
Jemand gewinnt den 2. Preis oder schaut zum 10ten Mal 425 Folge von BibBangTheory.\\
\\
Nat"urliche Zahlen $N$ sind \textbf{positive} Zahlen, die durch 0
bis 9 symbolisiert dargestellt werden.
Nat"urliche Zahlen k"onnen gepaart werden, indem man an den Zahlen 1 bis 9
weitere natürliche Zahlen anf"ugt (zum Bsp.: 12, 34, 22).
Bei der Aufstellung der nat"urlichen Zahlen ist wie in der mathelogie
"ublich eine einheitliche Form einzuhalten.
So kann/darf man bei einer Definition nicht einfach: 12, 1 33
schreiben !!! \\
Es ist zwar keine feste Regel daf"ur manifestiert, aber der "Ubersichtlichkeit
ist eine gewisse Disziplin der Ordnung nicht falsch. \\
\\
Die Menge der nat"urlichen \textbf{positiven} Zahlen werden wie folgt
definiert: \\
$N = \{ 0;1;2;3; \ldots \}$ \\
\\
Nat"urliche Zahlen sind nur innerhalb eines mehr oder weniger
gro"sen Bereichs abz"ahlbar, da sie unendlich sind. Jede nat"urliche Zahl
\textbf{n} hat immer einen unmittelbaren Nachfolger \textbf{n + 1} hat.

\section{ganze Zahlen}
\subsection{Einf"uhrung}
Das Ergebnis einer Addition oder einer Multiplikation zweier nat"urlicher Zahlen
is immer eine weitere nat"urliche Zahl. Man sagt, dass die Menge der n"urllichen Zahlen
bez"uglich der Addition und der Multiplikation in sich abgeschlossen sind.
Achtung: Dies gilt nicht für die Subtraktion und der Division !\\
Die Subtraktion zweier nat"urlichen Zahlen ist nur mit den im na"urlichen Wertebereich
der Zahlen 0 bis 9 möglich \textbf{und} wenn der Minuend größer ist als der Subtrahend.
Ist jedoch der Minuend größer als der Subtrahend, ist das Ergebnis der Subtraktion keine
nat"urliche Zahl mehr. Deshalb werden die nat"urlichen Zahlen um die negativen Zahlem
erg"angzt.\\
Ganze Zahlen $Z$ erweitern die nat"urlichen Zahlen $N$.

\section{rationale Zahlen}
\subsection{Einf"uhrung}
Die Menge der nat"urlichen Zahlen ist abgeschlossen bez"uglich der Addition und Multiplikation,
nicht jedoch bez"uglich der Subtraktion und Divison. Deshalb wurden die nat"urlichen Zahlen zu
den ganzen Zahlen erweitert. Die Menge der ganzen Zahlen ist nun auch abgeschlossen. Allerdings
ist die Division in den ganzen Zahlen nicht uneingeschr"ankt m"oglich.\\
Deshalb wird der Zahlenbereich nun ein zweites Mal erweitert, so dass er auch abgeschlossen ist.
Die neu einzuf"uhrenden Zahlen sind die Brüche, die als:\\
$\frac{a}{b}$ \\
geschrieben werden.

\section{gebrochene Zahlen}
\subsection{Einf"uhrung}
Gebrochene Zahlen $Q+$ oder $Q*$ sind eine Erweiterung der nat"urlichen
Zahlen $N$, die gestattet, uneingeschränkt zu dividieren.
Dazu werden nat"urliche Zahlen um den \textbf{negativen} Zahlenbereich von
$N$ sowie um Br"uche (rationale Zahlen $Q$) erg"anzt.\\
\\
Addition und Maltiplikation k"onnen als ''abgeschlossene Operationen'' betrachtet werden.
Die Summe zweier nat"urlicher Zahlen ergibt immer eine weitere nat"urliche Zahl:
\\
$3 + 5 = 8 \:\:\:\:\:\:\: | \:\: 3 \in \mathbb{N}; \: 5 \in \mathbb{N}; \: 8 \in \mathbb{N} \\
3 \: * 5 = 15 $
\\
\textbf{Minus und Division} gelten als ''nicht abgeschlossene Operationen''.
Die Differenz zweier nat"urlicher Zahlen muss nicht immer eine
nat"urliche ergeben:
\\
$3 - 5 = \: -2 \: \: \: | \: 3 \in \mathbb{N}; \: 5 \in \mathbb{N} \: -2 \notin \mathbb{N} \\
3 \: : 5 \: = \: 0,6 \: \: | \: 3 \in \mathbb{N}; \: 5 \in \mathbb{N} \:\: 0.6 \notin \mathbb{N}$\\
\\
\textbf{Bonus:}\\
Im Internet habe ich ein etwas verungl"ucktes Beispiel
zu unendliche N gefunden, das ich hier vorstellen, aber auch
Kommentieren will.\\
\\
- man denke sich ein kosmisches Hotel mit unendlich vielen
  Zimmern vor.\\
- das Hotel ist *voll* belegt.\\
- Nun kommt noch ein Gast.\\
\\
Frage: Kann er in einen voll belegten Hotel noch untergebracht
werden?
Antwort: = JA! \\

\begin{itemize}
        \item[-] da es unendlich viele Zimmer gibt, rückt jeder Gast nur ein
          Zimmer weiter und das \\
	  erste wird frei.
        \item[-] nach dem selben Prinzip k"onnen nat"urlich auch weitere 10 G"aste
          untergebracht werden.
\end{itemize}
        Kommentar von mir dazu:\\
        Der Sichtwinkel ist hierbei wichtig!
        Logisch ist es, wenn man von einen *vollen* Hotel spricht, das
        alle Betten belegt sind.
        Da aber der Begriff ''unendlich'', kein Ende oder *voll* definiert
        ist, k"onnen auch ''unendlich'' viele Betten/Zimmer bezogen werden.\\
        Anders ausgedr"uckt kann  die Zahl unendlich, kann an die Zahl
        unendlich angekn"upft werden, um wieder eine Menge von unendlich
        und nicht abz"ahlbaren Zahlen zu bekommen.

\section{reelle Zahlen}
\subsection{Einf"uhrung}
Die Menge der rationalen Zahlen ist abgeschlossen bez"uglich der
Addition, Subtraktion, Multiplikation und Division. Es gibt jedoch
Operationen, die aus den rationalen Zahlen herausf"uhren.
Eine dieser Operationen ist das Radizieren (wurzelziehen). Die Wurzeln der
meisten nat"urlichen Zahlen sind keine rationalen Zahlen. Dies werde ich sp"ater
für $\sqrt{2}$ zeigen.\\
Es existieren also Zahlen, die keine rationalen Zahlen sind. Um diese L"ucke zhu
schlie"sen, ben"otigt man eine Erweiterung des Zahlenbereichs. Die Zahlen, die
diese L"ucke auf der Zahlengerade beschreiben, nennt man \textbf{irrationale Zahlen}.
Sie lassen sich durch unendliche, nichtperiodische Dezimalbrüche darstellen.
Die rationalen und irrationalen $\mathbb{I}$ Zahlen ergeben die Menge der reellen Zahlen.
Sie wird mit $\mathbb{R}$ bezeichnet.
Die positiven reellen Zahlen werden entsprechend mit $\mathbb{R^+}$ und die
negativen mit $\mathbb{R^-}$ bezeichnet.\\
Bei den irrationalen Zahlen unterscheidet man zwischen algebraischen und transzendenten
Zahlen. Eine Zahl hei"st alghebraisch, wenn die L"osung einer Gleichung der Form:\\
$a_x^n + ... + a_2x^2 + a_1x + a_0 = 0$
\\
ist.\\
Zahlen, die nicht algebraisch sind, nennt man transzendent.
Transzendente Zahlen sind zum Beispiel die Kreiszahl $\pi = 3,1415...$ und die eulersche
Zahl $e = 2,71 ...$ .
Die eulersche Zahl spielt bei den Exponential- und Logarithmusfunktionen eine wichtige Rolle.

\subsection{Die Quadratwurzel aus 2 ist keine rationale Zahl}
Ich will nun zeigen, dass $\sqrt{2}$ keine rationale Zahl ist. Angenommen, dies w"are der Fall,
dann lie"se sich $\sqrt{2}$ als Bruch: $\sqrt{2} = \frac{a}{b}$ schreiben.\\
Man kann dabei ohne allgemeine Beschr"ankung annehmen, dass der Bruch eine Grunddarstellung ist.
Die Zahlen a und b sind also teilerfremd. Quadriert man die obige Gleichung, dann erh"alt man:\\
$2 = \frac{a^2}{b^2}$ .\\
Diese Gleichung kann man mit $b^2$ multiplizieren: $2 * b^2 = a^2$ . \\
Hieraus folgt, dass $a^2$ den Teiler 2 hat. Denn $a^2$ und $b^2$ sind ganze Zahlen.
Somit besitzt auch a den Teiler 2. Man kann deshalb die Zahl a durch a = 2*k ersetzen, wobei k
eine ganze Zahl ist.\\
Also muss auch $b^2$ den Teiler Teiler 2 haben. Dies steht aber im Widerspruch zu der
Voraussetzung, dass a und b teilerfremde Zahlen sind. Damit ist die Annahme, $\sqrt{2}$
sei eine rationale Zahl, falsch. Auf die gleiche Art kann man auch zeigen, dass die Wurzel
einer beliebigen Primzahl eine irrationale Zahl ist.

\section{komplexe Zahlen}
\subsection{Einf"uhrung}
Komplexe Zahlen $\mathbb{C}$ erweitern den Zahlenbereich des reellen Zahlenbereichs.\\
Beispiele solcher Zahlen sind: $i, 7 + 3i, 3 - 4i$\\
\\
Mit fortschreitender Erlangung von neuen Kenntnissen (z. Bsp. auch gepr"agt
von der Nutzung elektronischer Einheiten; womit ich
die Einf"uhrung des mathematische Bin"arsystems - $n^2$ andeuten will), wurde man
dadurch motiviert, komplexe Zahlen einzuf"uhren.\\
Man erkannte, das Gleichungen wie $x^2 = -1$ nicht l"osbar sind.
Da es nun aber auch die Zahl -1 (gesprochen: minus eins) in der
Zahlentheorie gibt, wurde eine \textbf{imaginäre Einheit} eingef"uhrt, die mit \textbf{i}
- als Definition: \textbf{ $ i^2 = -1 $ }  manifestiert wurde.
Mit komplexen Zahlen wurde somit das Problem behoben.

\section{Beispiel aus de.sci.mathematik}

Gegeben ist folgende Gleichung: \\ \\
$\sqrt[n]{(e^{(i \phi)}) = e^{i * (\frac{\phi}{n} + k * 2 * \frac{\pi}{n})} } $ mit  $ k = 0, 1, 2,..., n-1. $ \\

Hier die einfache Herleitung: \\
F"ur n = 2 und k = 0 aus der Gleichung: \\
\\
e entspricht der Eulerzahl 1 \\
i entspricht der imaginären Zahl: \: $ i^2 = -1 $ \\
dann ergibt sich aus e und i: \: $ 1^2 = -1 $ \\
$ \phi $ entsprich 1 * phi \\
$ \frac{\pi}{2} $ entsprich die H"alfte der Kreiszahl $\pi $: $ \frac{3.14}{2} = 1.57 $ \\
\\
1. $ \sqrt[2]{1^{-1 * \phi} = 1^{-1 * (\frac{\phi}{2} + 0 * 2 * \frac{\pi}{2})}} $ \\
2. $ \sqrt[2]{1^{-1 * \phi} = 1^{-1 * (\frac{\phi}{2} + 0 * 2 * 1.57)}} $ \\
3. $ \sqrt[2]{1^{-\phi} = \frac{1 * 2}{1} * \frac{\phi}{2}}  \: \: \: | \: \: 2 \: und \: 2 \: k"urzt \: sich \: weg. \: (\phi = 1)$\\
4. $ \sqrt[2]{1^{-1} = 1 * 1} $ \\
5. $ \sqrt[2]{1 = 1} = \sqrt[2]{1} = \sqrt[2]{1} $ \\
6. $ 1 = 1 $ \\
\\

Gegeben ist:\\ \\
1. $ -1 = (e^{2 * i * \pi})^{\frac{1}{2}} = (1*e^{2 * i * \pi})^{\frac{1}{2}} = (e^{2*i*\pi}*e^{2*i*\pi})^{\frac{1}{2}}=(e^{4*i*\pi})^{\frac{1}{2}} = 1 $ \\
\\
2. $ -1 = (e^{2 * i * \pi})^{\frac{1}{2}} = (e^{4 * i * \pi})^{\frac{1}{2}} $ \\
3. $ -1 = ({1^{2 * -1 * \pi}})^{\frac{1}{2}} = (1^{4 * -1 * \pi})^{\frac{1}{2}} $ \\
4. $ -1 = (1^{2 * -\pi})^{\frac{1}{2}} = (1^{4 * - \pi})^{\frac{1}{2}} $ \\
5. $ -1 = (-3.14)^{\frac{1}{2}} = (-3.14)^{\frac{1}{2}} $ \\
6. $ -1 = -1.57 = -1.57 $ \\
7. $ -1 = ((-1.57 \implies -1.57) = wahr = 1) = 1 $ \\
8. $ -1 = 1 = 1 $ \\
nun wird jedes Glied mit -1 multipliziert: \\
9. $ -1 * -1 = 1 \left\{
  \underbrace{
  \begin{array}{l}
   1 * -1 = -1 \\
   1 * -1 = -1
  \end{array}}_{\text{= 0, k"urzt sich weg}}
  \right\} = 1 * -1 = -1 $ \\
\\
Ergebnis: $ -1 \mapsto 1 = richtig\ ! $ \\

