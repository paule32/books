\chapter{Grundlagen - Mathematik}\label{Mathegrundlagen}
\section{Zahlensysteme}\label{Zahlensysteme}
In jedem Zahlensystem wird der Wert einer Ziffer durch ihre Stellung
innerhalb einer Zahl bestimmt. Die Grundlage eines jedem Zahlensystems ist
seine Basis.
\\
\begin{minipage}{1.5\textwidth}
\begin{tabular}{lrlcl}
Beispiel:  & m * $b^n$    & m     & = & Ziffernwert \\
           &              & b     & = & Basis \\
           &              & n     & = & Exponent \\
           &              & $b^n$ & = & Zahlenbasis \\
           &              &       &   & \\
           & $36_{(10)}$  &       &   & Zahlenbasis : 10 \\
\end{tabular}
\end{minipage}
\\
Zehner und Einer benennen den Stellenwert.\\
W"ahlt man dagegen einen andere Zahlenbasis, so "andert sich der Wert betr"achtlich.\\
\begin{minipage}{1.5\textwidth}
\begin{tabular}{lrlcrr}
           & $36_{(10)}$  &       &   &   Zahlenbasis : 16 & \\
Das egibt: &              &       &   &   $3 * 16^1$ =     & 48 \\
           &              &       &   & + $6 * 16^0$ =     &  6 \\
           &              &       &   & = $6 * 16^0$ =     & 54 \\
\end{tabular}
\end{minipage}

\subsection{Dualsystem}
Das duale System entspricht dem bin"aren Aufbau der elektronischen Datenverarbeitunh.
Es beruht auf der Basis 2 und ben"otigt nur die Ziffern 0 und 1.\\

\begin{minipage}{0.5\textwidth}
\begin{tabular}{llcr}
Beispiel:  & $10_{(2)}$ & = & $1 * 2^1$ = 2 \\
           &            & + & $0 * 2^0$ = 0 \\
           &            &   &           = 2 \\
\end{tabular}
\end{minipage}

\section{Umwandlung Dezimal - Dual}
Im ersten Fall wird die Dezimalzahl so lange durch die Basis 2 geteilt, bis das
Ergebnis Null erreicht ist. Die Reste der einzelnen Divisionen ergeben die
Dualzahl. Im zweiten Fall werden die Ziffernwerte mit den Stellenwerten der
Dualzahl multipliziert. Die Summe der Produkte ergibt die Dezimalzahl.
\\
\subsection{Dezimal nach Dual umwandeln}
\begin{minipage}{0.5\textwidth}
\begin{tabular}{rcrr}
29 : 2 & = & 14 + Rest 1 & \\
14 : 2 & = &  7 + Rest 0 & \\
 7 : 2 & = &  3 + Rest 1 & \\
 3 : 2 & = &  1 + Rest 1 & \\
 1 : 2 & = &  0 + Rest 1 &  =  $11101_{(2)}$ \\
\end{tabular}
\end{minipage}
\\
\subsection{Dual in Dezimalzahl umwandeln}
\begin{minipage}{0.5\textwidth}
\begin{tabular}{lrrcrl}
1 1 1 0 1 & = 1 * &  1 & = &  1 & \\
          &   0 * &  0 & = &  0 & \\
          &   1 * &  4 & = &  4 & \\
          &   1 * &  8 & = &  8 & \\
          &   1 * & 16 & = & 16 &  = $29_{(10)}$ \\
\end{tabular}
\end{minipage}

\section{Hexadezimalsystem}
In diesem System liegt die Basis 16 zu Grunde, d. h. es werden 16 Ziffern ben"otigt.
Die ersten zehn Ziffern entstammen dem Dezimalsystem, 0 bis 9, die folgenden sechs
Ziffern werden beginnend mit den ersten Buchstaben des Alphabets bezeichnet, also A bis F. \\

\begin{minipage}{0.5\textwidth}
\begin{tabular}{lcccccccccccccccc}
dual: & 0 & 1 & 2 & 3 & 4 & 5 & 6 & 7 & 8 & 9 & 10 & 11 & 12 & 13 & 14 & 15 \\
hex:  & 0 & 1 & 2 & 3 & 4 & 5 & 6 & 7 & 8 & 9 & A  & B  & C  & D  & E  & F
\end{tabular}
\end{minipage}

\subsection{Umwandlung Hex nach Dezimal}
Folgende Hexadezimalzahl ist gegeben: 2AE\\
\\
\begin{minipage}{0.5\textwidth}
\begin{tabular}{lcrccrrl}
2 & = &  2 & * & $16^2$ & = & $512_{(10)}$ \\
A & = & 10 & * & $16^1$ & = & $160_{(10)}$ \\
E & = & 14 & * & $16^0$ & = & $ 14_{(10)}$ &  = $686_{(10)}$ \\
\end{tabular}
\end{minipage}

\section{Zeichen und Symbole}

\begin{minipage}{0.5\textwidth}
    \centering
    \captionof{figure}{Test figure}
    \legendezbereiche \bildzrange
    \label{fig:zahlenbereiche}
\end{minipage}

